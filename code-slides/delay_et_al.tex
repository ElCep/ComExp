% 	Name		:: 	sthlm Beamer Theme  HEAVILY based on the hsrmbeamer theme (Benjamin Weiss)
%	Author		:: 	Mark Hendry Olson (mark@hendryolson.com)
%	Created		::	2013-07-31
%	Updated		::	June 18, 2015 at 08:45
%	Version		:: 	1.0.2
%	Email		:: 	hendryolson@gmail.com
%	Website		:: 	http://v42.com
%
% 	License		:: 	This file may be distributed and/or modified under the
%                  	GNU Public License.
%
%	Description	::	This presentation is a demonstration of the sthlm beamer
%					theme, which is HEAVILY based on the HSRM beamer theme created by Benjamin Weiss
%					(benjamin.weiss@student.hs-rm.de), which can be found on GitHub
%					<https://github.com/hsrmbeamertheme/hsrmbeamertheme>.


%-=-=-=-=-=-=-=-=-=-=-=-=-=-=-=-=-=-=-=-=-=-=-=-=
%
%        LOADING DOCUMENT
%
%-=-=-=-=-=-=-=-=-=-=-=-=-=-=-=-=-=-=-=-=-=-=-=-=

\documentclass[newPxFont]{beamer}
\usetheme{sthlm}
%\usecolortheme{sthlmv42}

%-=-=-=-=-=-=-=-=-=-=-=-=-=-=-=-=-=-=-=-=-=-=-=-=
%        LOADING PACKAGES
%-=-=-=-=-=-=-=-=-=-=-=-=-=-=-=-=-=-=-=-=-=-=-=-=
\usepackage[utf8]{inputenc}
\usepackage[T1]{fontenc}

%\usepackage{chronology}
\usepackage{chronosys}
\usepackage{subfigure}

\newcommand{\tabitem}{%
  \usebeamertemplate{itemize item}\hspace*{\labelsep}}


%\renewcommand{\event}[3][e]{%
%  \pgfmathsetlength\xstop{(#2-\theyearstart)*\unit}%
%  \ifx #1e%
%    \draw[fill=black,draw=none,opacity=0.5]%
%      (\xstop, 0) circle (.2\unit)%
%      node[opacity=1,rotate=45,right=.2\unit] {#3};%
%  \else%
%    \pgfmathsetlength\xstart{(#1-\theyearstart)*\unit}%
%    \draw[fill=black,draw=none,opacity=0.5,rounded corners=.1\unit]%
%      (\xstart,-.1\unit) rectangle%
%      node[opacity=1,rotate=45,right=.2\unit] {#3} (\xstop,.1\unit);%
%  \fi}%

%-=-=-=-=-=-=-=-=-=-=-=-=-=-=-=-=-=-=-=-=-=-=-=-=
%        BEAMER OPTIONS
%-=-=-=-=-=-=-=-=-=-=-=-=-=-=-=-=-=-=-=-=-=-=-=-=

%\setbeameroption{show notes}

%-=-=-=-=-=-=-=-=-=-=-=-=-=-=-=-=-=-=-=-=-=-=-=-=
%
%	PRESENTATION INFORMATION
%
%-=-=-=-=-=-=-=-=-=-=-=-=-=-=-=-=-=-=-=-=-=-=-=-=

\title{ComExp}
\subtitle{Comment pousser l'accompagnement jusqu'à l'exploration des modèles à base d'agents.}
%\date{\small{\jobname}}
%\date{\today}
\author{\texttt{E. Delay}\\
avec une aimable stimulation de \texttt{R. Reuillon}, \texttt{P. Chapron} et \texttt{M. Leclaire}}
\institute{CIRAD -- UMR SENS}

\hypersetup{
pdfauthor = {E. DELAY},
pdfsubject = {Seminaire SENS},
pdfkeywords = {Communs, Simulation, solidarité},
pdfmoddate= {\pdfdate},
pdfcreator = {}
}

\begin{document}

%-=-=-=-=-=-=-=-=-=-=-=-=-=-=-=-=-=-=-=-=-=-=-=-=
%
%	TITLE PAGE
%
%-=-=-=-=-=-=-=-=-=-=-=-=-=-=-=-=-=-=-=-=-=-=-=-=


\maketitle

%\begin{frame}[plain]
%	\titlepage
%\end{frame}

%-=-=-=-=-=-=-=-=-=-=-=-=-=-=-=-=-=-=-=-=-=-=-=-=
%
%	TABLE OF CONTENTS: OVERVIEW
%
%-=-=-=-=-=-=-=-=-=-=-=-=-=-=-=-=-=-=-=-=-=-=-=-=
% \section*{Une boussole ?}
% \begin{frame}{Overview}
% % For longer presentations use hideallsubsections option
% \tableofcontents[hideallsubsections]
% \end{frame}

%-=-=-=-=-=-=-=-=-=-=-=-=-=-=-=-=-=-=-=-=-=-=-=-=
%	FRAME: INTRODUCTION
%-=-=-=-=-=-=-=-=-=-=-=-=-=-=-=-=-=-=-=-=-=-=-=-=

\section{Introduction :\\ Au commencement il y avait ComMod}
  \begin{frame}[c]{L'égalité comme pré-requis}
    \vspace{-1cm}
    \begin{columns}[onlytextwidth,T]
      \column{\dimexpr\linewidth-30mm-5mm}
          \begin{itemize}
            \item Le \textit{``maître ignorant''} de Rancière (2003) qui accompagne : <<\emph{contrairement à ce que laissent penser nos positions sociales, nous sommes égaux, voyons ce que nous pouvons en faire}>>
            \item Le diplomate de Morizot (2020) cherche les axes de mobilisation, car <<\emph{une fois qu’on a circulé parmi les points de vue, on sent que certains n’ont pas la légitimité qu’ils réclament.}>>
          \end{itemize}

          \small{
              \begin{alertblock}{\textsc{Une posture philosophique}}
                Entre \textsc{Rancière} et \textsc{Morizot}, on a les composantes de la posture des acteurs du vivre ensemble.
              \end{alertblock}
            }
      \column{30mm}
      \vspace{0.5cm}
            \includegraphics[width=3cm]{img/Ranciere.jpg}\\
            \includegraphics[width=3cm]{img/morizot.jpg}
    \end{columns}
  \end{frame}

  \begin{frame}[c]{L'équité : un objectif ?}
    \vspace{-1cm}
    l'approche par les communs a besoin d'un préalable, qui est l'égalité entre les participants.
    \begin{figure}
      \includegraphics[height=7cm]{img/commun_egalite_equite.png}
    \end{figure}
  \end{frame}

  \begin{frame}[c]{Le décalage prométhéen et l'urgence environnementale}
    \vspace{-1cm}
    \begin{columns}[onlytextwidth,T]
      \column{\dimexpr\linewidth-30mm-5mm}
      \begin{itemize}
        \item Dans \textit{"l'obsolescence de l'Homme"} Günther Anders (1956) propose la notion de décalage prométhéen,
        \item c'est-à-dire le fait que les capacités de fabrication dépassent de très loin nos possibilités de représentation
      \end{itemize}

      \small{
        \begin{alertblock}{\textsc{Une hypothese forte}}
          Accompagner les participants, c'est réduire le décalage prométhéen, et permettre le passage à l'action.
        \end{alertblock}
      }
      \column{30mm}
      \includegraphics[height=5cm]{img/promethee.jpg}
    \end{columns}
  \end{frame}

%-=-=-=-=-=-=-=-=-=-=-=-=-=-=-=-=-=-=-=-=-=-=-=-=
%	FRAME: ComExp qu'est-ce que c'est ? 
%-=-=-=-=-=-=-=-=-=-=-=-=-=-=-=-=-=-=-=-=-=-=-=-=

\section{ComExp\\ Un étage de plus a la fusé ComMod}


\begin{frame}[c]{ComExp c'est ComMod}
  \vspace{-1cm}
  Avec des ateliers de co-construction
  \begin{center}
  \includegraphics[width=7cm]{img/atelier_niakhar.jpg}
  \end{center}
\end{frame}

\begin{frame}[c]{ComExp c'est ComMod}
  \vspace{-1cm}
  Avec des ARDI en \texttt{.dot}
  \begin{center}
  \includegraphics[width=8cm]{img/pardi_fdp.png}
  \end{center}
\end{frame}

\begin{frame}[c]{ComExp c'est ComMod}
  \vspace{-1cm}
  Avec des modèles de simulations
  \begin{center}
  \includegraphics[width=9cm]{img/modelSimu.JPG}
  \end{center}
\end{frame}

\begin{frame}[c]{Mais ComMod, avec un petit truc en plus}
  \vspace{-1cm}
  \small Un usage massif au calcul en essayant d'échaper a la malédiction des dimensions.
\begin{center}
 %\includegraphics[width=6cm]{img/OpenMOLE-Banner.png}\\
 \includegraphics[width=8cm]{img/CurseOfDimensionality.png}
\end{center}
\end{frame}


%-=-=-=-=-=-=-=-=-=-=-=-=-=-=-=-=-=-=-=-=-=-=-=-=
%	FRAME: ComEXp - permet de changer le point de vue
%-=-=-=-=-=-=-=-=-=-=-=-=-=-=-=-=-=-=-=-=-=-=-=-=

\section{ComExp\\ pour poser d'autres questions}


%-=-=-=-=-=-=-=-=-=-=-=-=-=-=-=-=-=-=-=-=-=-=-=-=
%	FRAME: ComEXp - pour quoi faire ? 
%-=-=-=-=-=-=-=-=-=-=-=-=-=-=-=-=-=-=-=-=-=-=-=-=

\section{ComExp\\ pour quoi faire ?}

\begin{frame}[c]{Chercher les bordures}
  \vspace{-1cm}
  \begin{columns}[onlytextwidth,T]
    \column{\dimexpr\linewidth-30mm-5mm}
        \begin{itemize}
          \item Des algorythme génétiques
          \item Optimiser VS chercher la diversité
        \end{itemize}
        \begin{alertblock}{\textsc{Prendre appui pour changer}}
          C'est parce qu'on peut prendre appui sur les bords du système, rediscuter les contrôles, donner du sens aux intervention qu'on peut transformer les rapports de forces et la connaissance (F. Julien, p.15, 2009).
         \end{alertblock}
    \column{30mm}
    \vspace{0.5cm}
    \includegraphics[width=7cm]{img/pse.png}
  \end{columns}
\end{frame}

\begin{frame}[c]{Le possible et le plausible}
  \vspace{-1cm}
  \begin{itemize}
    \item \textbf{Possible}\footnote{Définition du Centre de Ressources Textuelles et Lexicales \url{https://cnrtl.fr/definition/possible}, consulté le 11 décembre 2024.} : Adj. Qui remplit les conditions nécessaires pour être, exister, se produire sans que cela implique une réalisation effective ou que l'on sache si cette réalisation a été, est ou sera effective.

    \item \textbf{Plausible}\footnote{Définition du Centre de Ressources Textuelles et Lexicales \url{https://cnrtl.fr/definition/plausible}, consulté le 11 décembre 2024.} : Que l'on peut admettre ou croire parce que vraisemblable.
\end{itemize}
\begin{center}
 \includegraphics[width=9cm]{img/possiblePlausible.png}
\end{center}
\end{frame}

\begin{frame}[c]{Pour anticiper}
  \vspace{-1cm}
\begin{center}
  \includegraphics[width=5cm]{img/dundiModel_explo.png}
  \includegraphics[width=6cm]{img/dundiModel.png}
\end{center}
\end{frame}

\begin{frame}[c]{Aller chercher le réseau soc. tech.}
  \vspace{-1cm}
\begin{center}
  \includegraphics[width=9cm]{img/fig-construction-elements-jeu-beneficiaire_eng}
\end{center}
\end{frame}


\begin{frame}[c]{Traquer les changement de régimes}
  \vspace{-1cm}
  \begin{center}
    \includegraphics[width=7cm]{img/2024mathias.png}\\
    \includegraphics[width=7cm]{img/m0_pse_fatisfaction_mathias_biomass.png}
   \end{center}
\end{frame}

%-=-=-=-=-=-=-=-=-=-=-=-=-=-=-=-=-=-=-=-=-=-=-=-=
%	FRAME: OLDY
%-=-=-=-=-=-=-=-=-=-=-=-=-=-=-=-=-=-=-=-=-=-=-=-=

\section{Un genre de conclusion ?}

\begin{frame}[c]{Vers des sci. transformatives ?}
  \vspace{-1cm}
  \begin{columns}[onlytextwidth,T]
    \column{\dimexpr\linewidth-30mm-5mm}
    \begin{enumerate}
        \item On retrouve évidement ici la vision des capabilité de \textsc{Sen} (1999)
        \item le lien avec la vision de l’accompagnement de \textsc{Rancière} (2003)
        \item La réduction du décalage prométhéen de \textsc{Anders} (1956) pouvoir passer a l'action
        %\item Si on accepte le fait que "l'usage est une fin en soi" $\rightarrow$ on garde une posture de domination
    \end{enumerate}
    Le processus transformatif a pour objectif de rendre libre les participants. La relation n°1 doit permettre de mesure le degré de liberté qu’on peut obtenir par l’action.
    \includegraphics[width=7.5cm]{img/Drawing_jpm}
    \column{30mm}
    \begin{figure}
      \includegraphics[width=3.9cm]{img/joker.jpg}
    \end{figure}
  \end{columns}
\end{frame}

%-=-=-=-=-=-=-=-=-=-=-=-=-=-=-=-=-=-=-=-=-=-=-=-=
%	FRAME: MERCI DE VOTRE ATTENTION
%-=-=-=-=-=-=-=-=-=-=-=-=-=-=-=-=-=-=-=-=-=-=-=-=
{
\usebackgroundtemplate{
\includegraphics[width=\paperwidth]{img/arroseur_diohine.JPG}}%
\begin{frame}
  \vspace{-1em}
  \begin{minipage}[t][.8\textheight]{\textwidth}
    \color{\cnGrey}{\LARGE{Merci de votre attention}}

    \vfill

  %\hfill \small{Photo credit : Thomas m-louis. sur \includegraphics[height=0.55cm]{img/flickr_logo}}
  \end{minipage}
  \vspace{-3.5em}
  \centering
	You can find this presentation on github\includegraphics[height=0.85cm]{img/github}

\end{frame}
}


\end{document}